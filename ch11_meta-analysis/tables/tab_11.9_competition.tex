
\begin{table}[!htbp] \centering 
  \caption{Effect of information and electoral competition on vote choice} 
  \label{competition} 
\begin{tabular}{@{\extracolsep{1pt}}lcccc} 
\\[-1.8ex]\hline 
\hline \\[-1.8ex] 
 & \multicolumn{4}{c}{Incumbent vote choice} \\ 
\cline{2-5} 
& \multicolumn{2}{c}{Low competition} &\multicolumn{2}{c}{High competition} \\
\cline{2-5}
 & Good news & Bad news & Good news & Bad news \\ 
\\[-1.8ex] & (1) & (2) & (3) & (4)\\ 
\hline \\[-1.8ex] 
 Treatment & 0.009 & $-$0.043 & 0.004 & 0.015 \\ 
  & (0.022) & (0.031) & (0.030) & (0.037) \\ 
  & & & & \\ 
\hline \\[-1.8ex] 
Control mean & 0.342 & 0.414 & 0.392 & 0.294 \\ 
RI $p$-values & 0.692 & 0.272 & 0.912 & 0.757 \\ 
Covariates & No & No & No & No \\ 
Observations & 1,450 & 1,433 & 1,113 & 1,307 \\ 
R$^{2}$ & 0.221 & 0.231 & 0.240 & 0.128 \\ 
\hline 
\hline \\[-1.8ex] 
\end{tabular} 
\begin{flushleft}\textit{Note:} The table reports results of whether the treatment had different effects in constituencies with low or high levels of electoral competition (MPAP measure M25). We pool Benin, Brazil, Mexico, and Uganda 1. Regressions include randomization block fixed effects; standard errors are clustered at the level of treatment assignment. $^*$ $p<0.05$; $^{**}$ $p<0.01$; $^{***}$ $p<0.001$ \end{flushleft}
\end{table} 
