
\begin{table}[htb] \centering 
  \caption{Private vs Public Information: Effect of good news on incumbent vote choice} 
  \label{pvt_pub_good} 
\begin{tabular}{@{\extracolsep{5pt}}lcccc} 
\\[-1.8ex]\hline 
\hline \\[-1.8ex] 
 & \multicolumn{4}{c}{Incumbent vote choice, good news} \\ 
\cline{2-5} 
 & Overall & Benin & Mexico & Uganda 1 \\ 
\\[-1.8ex] & (1) & (2) & (3) & (4)\\ 
\hline \\[-1.8ex] 
 Private information & $-$0.008 & 0.012 & $-$0.029 & 0.008 \\ 
  & (0.023) & (0.044) & (0.043) & (0.027) \\ 
  & & & & \\ 
 Public information & 0.055$^{*}$ & 0.146$^{**}$ & $-$0.002 & 0.019 \\ 
  & (0.022) & (0.047) & (0.041) & (0.023) \\ 
  & & & & \\ 
\hline \\[-1.8ex] 
Control mean & 0.356 & 0.439 & 0.498 & 0.186 \\ 
F-test $p$-value & 0.018 & 0.006 & 0.598 & 0.708 \\ 
Covariates & No & No & No & No \\ 
Observations & 2,962 & 776 & 784 & 1,402 \\ 
R$^{2}$ & 0.192 & 0.189 & 0.088 & 0.068 \\ 
\hline 
\hline \\[-1.8ex] 
\end{tabular} 
\begin{flushleft}\textit{Note:} The table reports results of the effect of good news about the incumbent on vote choice, depending on whether voters received this information in private or public settings. We pool Benin, Mexico, and Uganda 1. Regressions include randomization block fixed effects and standard errors are clustered at the level of treatment assignment. $^*$ $p<0.05$; $^{**}$ $p<0.01$; $^{***}$ $p<0.001$ \end{flushleft}
\end{table} 
