
\begin{table}[!htbp] \centering 
  \caption{Relationship between evaluation of  politician effort and honesty with vote choice [unregistered analysis]} 
  \label{m1_effort_honesty} 
\begin{tabular}{@{\extracolsep{1pt}}lcccc} 
\\[-1.8ex]\hline 
\hline \\[-1.8ex] 
 & \multicolumn{4}{c}{Incumbent vote choice} \\ 
\cline{2-5} 
 & \multicolumn{2}{c}{Good news} & \multicolumn{2}{c}{Bad news} \\ 
\cline{2-5}
\\[-1.8ex] & (1) & (2) & (3) & (4)\\ 
\hline \\[-1.8ex] 
 Effort & 0.052$^{***}$ &  & 0.066$^{***}$ &  \\ 
  & (0.006) &  & (0.006) &  \\ 
  & & & & \\ 
 Dishonesty &  & $-$0.054$^{***}$ &  & $-$0.026$^{***}$ \\ 
  &  & (0.005) &  & (0.005) \\ 
  & & & & \\ 
\hline \\[-1.8ex] 
Covariates & No & No & No & No \\ 
Observations & 11,040 & 11,452 & 10,190 & 10,943 \\ 
R$^{2}$ & 0.229 & 0.217 & 0.282 & 0.266 \\ 
\hline 
\hline \\[-1.8ex] 
\end{tabular} 
\begin{flushleft}\textit{Note:} The table reports the effects of information and the credibility of the information source on voter's perception of how hard-working (MPAP measure M5) and dishonest (MPAP measure M6) the incumbent politician is. We pool Benin, Burkina Faso, Uganda 1, and Uganda 2 in columns (1) and (3), and Benin, Burkina Faso, Mexico, and Uganda 2 in columns (2) and (4). Results exclude non-contested seats and include vote choice for LCV councilors as well as chairs in the Uganda 2 study. Regressions include randomization block fixed effects; standard errors are clustered at the level of treatment assignment. $^*$ $p<0.05$; $^{**}$ $p<0.01$; $^{***}$ $p<0.001$ \end{flushleft}
\end{table} 
