\begin{table}[h!]
\caption{Differential attrition}
\centering
\begin{tabular}{rrlrrlr}
  \hline& \multicolumn{3}{c}{Vote Choice}&\multicolumn{3}{c}{Voter Turnout} \\
 & Estimate & Std. Error & $p$-value & Estimate & Std. Error & $p$-value \\ 
  \hline
Treatment & 0.00 & (0) & 0.57 & 0.00 & (0) & 0.71 \\ 
   \hline
F-stat & \multicolumn{3}{c}{13.78}&\multicolumn{3}{c}{15.26} \\
Pr(F) & \multicolumn{3}{c}{0.39}&\multicolumn{3}{c}{0.29} \\ \hline \hline
\end{tabular}
\begin{flushleft}\textit{Note:} Table shows the effect size of treatment on data missingness in incumbent vote choice and voter turnout across the entire sample. Pr(F) shows the probability of rejecting the null that none of the covariates is differentally determining attrition across treatment and control conditions. All regressions include block fixed effects, standard errors clustered at the level of assingment and inverse propensity weights, and all countries are weighted equally.$^*$ $p<0.05$; $^{**}$ $p<0.01$; $^{***}$ $p<0.001$. \end{flushleft}
\end{table}
